%%%%%%%%%%%%%%%%%%%%%%%%%%%%%%%%%%%%%%%%%
% Medium Length Professional CV
% LaTeX Template
% Version 3.0 (December 17, 2022)
%
% This template originates from:
% https://www.LaTeXTemplates.com
%
% Author:
% Vel (vel@latextemplates.com)
%
% Original author:
% Trey Hunner (http://www.treyhunner.com/)
%
% License:
% CC BY-NC-SA 4.0 (https://creativecommons.org/licenses/by-nc-sa/4.0/)
%
%%%%%%%%%%%%%%%%%%%%%%%%%%%%%%%%%%%%%%%%%

%----------------------------------------------------------------------------------------
%	PACKAGES AND OTHER DOCUMENT CONFIGURATIONS
%----------------------------------------------------------------------------------------

\documentclass[
	%a4paper, % Uncomment for A4 paper size (default is US letter)
	11pt, % Default font size, can use 10pt, 11pt or 12pt
]{resume} % Use the resume class

\usepackage{ebgaramond} % Use the EB Garamond font

%------------------------------------------------

\name{LI Shangxuan} % Your name to appear at the top

% You can use the \address command up to 3 times for 3 different addresses or pieces of contact information
% Any new lines (\\) you use in the \address commands will be converted to symbols, so each address will appear as a single line.

% \address{A1, 16F, 83--97 Nathan Rd., TST} % Main address

% \address{123 Pleasant Lane \\ City, State 12345} % A secondary address (optional)

\address{(+852) 5482-6196 \\ 22626433@life.hkbu.edu.hk} % Contact information

%----------------------------------------------------------------------------------------

\begin{document}


\begin{rSection}{Education}

    \textbf{Hong Kong Baptist University} \hfill 2022 --- Present \\
    Associate of Science\\
    (Concentration Studies: Psychology)\\
    % Minor in Linguistics \smallskip \\
    % Member of Eta Kappa Nu \\
    % Member of Upsilon Pi Epsilon \\
    cGPA: 3.66/4.00

    \textbf{Dongguan Middle School Songshan Lake School} \hfill 2019 --- 2022 \\
    High School Diploma \\
    (Science Course Set)\\
    National College Entrance Examination: 543/750

\end{rSection}


\begin{rSection}{Experience}

    % \begin{rSubsection}{Personal Research Project: Freshman Adjustment in Cross-Cultural Aspect}{2023 --- Present}{}{}
    %     \item Group differences of freshman adjustment
    %     \item Relationships between cultural variables and adjustment outcomes.
    % \end{rSubsection}

    \begin{rSubsection}{Anti-Drug Animation Festival}{2023}{}{}
        \item Healthy Transition II program driven by Hong Kong Federation of Youth Groups
        \item Involved in creating and using animations as a medium to convey the message
        \item Enhanced skills in storytelling, graphic design, and animation
        \item Learnt the use of multimedia in the advocation of crucial social causes
        \item Raised awareness about the detrimental impacts of drug abuse on individuals and society
    \end{rSubsection}

    \begin{rSubsection}{Mental Health Gate Keeper}{2022 --- 2023}{}{}
        \item Certificated after Mental Health First Aid Training
        \item Participated in the psychology exhibition of wellness week
        \item Engaged in meaningful dialogues and discussions centered around mental well-being
        \item Exposed to the latest research, techniques, and tools in the field of psychology
    \end{rSubsection}

    \begin{rSubsection}{Student Suicide Prevention \& Secondary Students' Mental Health Enhancement Professional Training}{2022}{}{}
        \item Understood different kinds of mental health distress
        \item Equipped with knowledge  for recognizing early signs of suicide and techniques for risk evaluation
        \item Familiered with structured early intervention skills and tools before professional help
    \end{rSubsection}

    \begin{rSubsection}{Olympiad in Informatics}{2018 --- 2022}{}{}
        \item Joined National Olympiad in Informatics in Provinces every year and won second prize
        \item Learned programming skills, especially algorithm and data structure knowledge
        \item Developed thinking and problem-solving ability
        \item Accumulated computer using skills
    \end{rSubsection}

\end{rSection}


\begin{rSection}{Technical Strengths}

    \begin{tabular}{@{} >{\bfseries}l @{\hspace{6ex}} l @{}}
        Academic Skills  & SPSS, \LaTeX, Zotero                                        \\

        Auxiliary Skills & C++, Git, Office                                            \\
        Languages        & English (Fluent), Cantonese (Elementary), Mandarin (Native)
    \end{tabular}

\end{rSection}


% \begin{rSection}{Self-evaluation}
%     \textbf{Academic Performence} \\
%     Passionate about Psychology, strictly demanding my assignments according to academic standards.
%     Want to accumulate more research-related skills for further studies.

%     \textbf{Research Experience } \\
%     Passionate about research, enjoying inventing and discovering new things. Positively applied learned research methods in daily studies and now working on a research project in Psychology.

%     \textbf{Group Work} \\
%     Often lead team members, completing research tasks efficiently and rigorously.

%     % \textbf{Other Interests}\\
%     % Literary, Photography, Visiting Exhibitions
% \end{rSection}

%----------------------------------------------------------------------------------------
%	EXAMPLE SECTION
%----------------------------------------------------------------------------------------

%\begin{rSection}{Section Name}

%Section content\ldots

%\end{rSection}

%----------------------------------------------------------------------------------------

\end{document}
