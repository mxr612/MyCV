\documentclass{article}

\usepackage{fontspec,xeCJK}
% \setmainfont{SmileySans-Oblique}
\setCJKmainfont{SmileySans-Oblique}

\usepackage{mathcmd}

\begin{document}

% \section{申请动机}

% 之所以走心理学路线,是因为我意识到

作为一个高Openness的人,我很乐意认识不同的人,并在交友中偏好倾听并尝试理解其他人。
我认为人们总有不同的原因成为现在的自己,不论是生理还是环境原因。学习心理学给我很大的愉悦,在不断了解自己和他人的过程中保持开放、接纳、不评判。

% \section{未来\&发展}

我希望将自己奉献给心理学,努力攻读心理学本科和研究生学位。毕业后也希望留在学术机构中继续为推进心理学发展尽心尽力。
我对心理学非常有热情,尤其是文化心理学方向。
在学习过程中,我发现现今的很多心理学研究的结果多多少少受到采样范围的限制。更具体地说,造成这种差异的原因主要是因为大多数研究旨在更多地了解西方人,而招募的参与者也主要是西方人。东西方心理数据的巨大差距,导致缺乏实用的、有影响力的心理学发现来具体解释东方国家存在的心理现象。
文化作为一种环境,其实很大程度上影响了人的思考和行动,然而一直处于缺乏详细讨论的状态。现今国际冲突与地域矛盾越发普遍,增进人们互相理解是全世界范围内的一个非常重要的课题。

% \section{补充说明}

在过去一年中,学习之余我也积极参加了一些香港本地的心理学相关活动,如心理健康急救课程、心理健康社团等。
在副学士学习中,我对学术非常用心。在完成个人功课和小组作业的过程中,我积极阅读文献并尽己所能用最高的学术标准要求自己,保证每一份作业的创新性和严谨性。
同时也学习了一些有助于学术工作的软件和技能,如论文管理工具和 \LaTeX 文本编辑。

% \section{特殊要求}

\end{document}