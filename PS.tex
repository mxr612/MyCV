\documentclass[]{article}

\title{Personal Statement}
\author{LI Shangxuan}

\begin{document}

\maketitle

Having originated from a land known for its rich tapestry of cultures and traditions, I've been nurtured in an environment that cherishes diversity and values the uniqueness of individual perspectives.
As a person characterized by high levels of openness, I embrace the beauty in diversity, longing to explore the myriad complexities of human behavior and the psychological underpinnings that define our existence.
My fascination with understanding human behavior and the underlying psychological mechanisms has always resonated deeply within me.
As someone who naturally gravitates towards high Openness, I find immense satisfaction in interacting with diverse groups of people, striving to comprehend their varied perspectives.
This intrinsic curiosity has led me down the path of psychology, fostering a commitment to explore cultural psychology in particular.

Today's world is an amalgamation of numerous cultures, each with its unique thought processes, customs, and ideologies.
However, most psychological studies predominantly focus on Western viewpoints.
This limitation, which I noticed during my undergraduate studies, raised critical questions: How do we ensure that psychological research is universally applicable? How do we bridge the knowledge gap between Western and Eastern psychologies? These questions prompted me to undertake a descriptive study titled ``Cross-cultural Comparative Research of Freshman Year Adaptation''.
This research, which spanned three months and remains a work in progress, focuses on understanding how freshmen from different parts of the world, including China, the USA, and Europe, adapt to new academic environments.
Our study considers various factors such as adolescent growth, socio-economic status, and differences between internal and external coping mechanisms.
The aim is to offer a comprehensive understanding of how cultural backgrounds influence students' adaptation to new environments.

While I am driven by my personal interest in cultural psychology, my dedication is further underpinned by my academic journey.
Studying at the Hong Kong Baptist University, I was exposed to a vast array of subjects within psychology.
From biological to social psychology, I delved deep into each subject, striving to assimilate the theoretical knowledge while seeking real-world applications.
This dedication is reflected in my cGPA of $3.66/4.00$.
Outside the confines of a classroom, I actively participated in various psychology-related activities in Hong Kong.
My involvement in initiatives such as the Mental Health First Aid Course and Mental Health Club underscores my commitment to understanding and enhancing psychological well-being at the grassroots level.

The juxtaposition of my academic pursuits and extracurricular activities has allowed me to appreciate the importance of cultural psychology in our increasingly interconnected world.
My experiences have only reaffirmed my belief that culture, as an intrinsic environment, profoundly impacts the way we think and act.
In a world where international tensions are on the rise, fostering mutual understanding is paramount.
And I believe that cultural psychology can be the bridge to facilitate this understanding.
While I am ardent about studying and understanding psychological phenomena, I am equally passionate about sharing this knowledge.
My aspiration is not only to further my studies with an undergraduate and postgraduate degree in psychology but also to remain within academia, advancing the field through research and teaching.

My inclination towards cultural psychology is spurred by the observation of a significant gap in psychological data between Eastern and Western demographics.
A substantial part of psychological studies predominantly focus on Western participants, leading to a lack of practical findings relevant to Eastern countries.
In this global age where international and regional conflicts burgeon, fostering mutual understanding through detailed discussions on cultural influences on psychology is paramount.
It is my aspiration to contribute significantly to bridging this gap through extensive research, focusing on bringing to light the unexplored facets of psychology in Eastern cultures.
Post-graduation, my vision is to continue contributing to the academic sphere, advancing the field of psychology through relentless research and exploration, and imparting acquired knowledge to the coming generations.

Although I approached professors for mentorship and found hesitance, it has not deterred me.
I remain steadfast in my goal, investing time into reading numerous papers and furthering my understanding.
The passion I harbor for psychology is not just confined to my academic and research interests; it defines who I am and the contributions I aim to make to the world.
My diverse skill set, ranging from my proficiency in tools like SPSS and \LaTeX\ typesetting language, to my ability to communicate fluently in English and Mandarin, positions me as an ideal candidate for a graduate program at your esteemed university.
My experiences, both academic and extracurricular, have shaped my perspective, enabling me to approach challenges with an analytical mindset.

In conclusion, the path of psychology is not just a career choice for me; it's a vocation.
With my zeal for cultural psychology and my commitment to academic rigor, I am confident in my ability to contribute significantly to the ongoing research at your university.
I am keen on joining your esteemed institution, not just as a student, but as an individual who can bring a unique perspective and make a lasting impact.

\end{document}